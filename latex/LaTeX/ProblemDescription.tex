\chapter{Problem Description}
The main goal of this dissertation is to provide a way to carry out a quantitative content analysis of the corpus of published texts within the EU’s political context. This process however, is not as straightforward as it may seem and an array of issues pop up.

One first significant difficulty has to do with the tremendous amount of relevant actors in this field. In order for an association, in this case, APPLiA, to be able to position itself with respect to a particular policy proposition, it is first and foremost necessary to identify each and every actor that could potentially influence the proposition. This group of relevant actors is made up of not only other lobby groups, both national and European ones, but also different NGOs and consumer rights organisations. Of course, even certain companies, especially large companies, and individuals can adopt a position on legislative matters. Mapping out the actors involved in EU legislative processes has been made easier through the introduction of the Have Your Say platform by the EU Commission, which enables citizens, corporations and lobby groups to express their opinion at all stages of the policy making process. The platform itself is discussed in more detail in section 3 on the methodology of this dissertation.

Secondly, besides identifying the relevant actors in a certain document, different actors will inevitably focus on different aspects of a proposal. Not only can a proposal coming from the EU Commission group multiple measures, even when only one measure is included, an action by the Commission will trigger different effects in almost all cases. Therefore, it is important to categorise which issues regarding a specific policy proposition are of significant consequence to which actors. In other words, it is necessary to determine for each actor which pillars of the regulation they focus on. So, the challenge when applying quantitative content analysis to the study of EU lobbying is analysing a policy space that has more than one dimension. Current applications of quantitative text analysis researching EU lobbying have used the complete text of a position document to estimate one policy dimension only \autocite{kluver2009}. Yet, the realities of EU policy making in general, and those of the EC consultations in particular, speak of a policy space that has more than one dimension \autocite{bunea2015}.

Thirdly, for each actor, the ultimate goal of the analysis is not only to distinguish the different aspects that are relevant for this organisation, but also to predict the policy outcome that is prefered by the actor. While for some proposals the position of the actor is clear cut, this will not necessarily be evident in more multifaceted proposals. For a simple proposal, a position paper might unambiguously state whether they support or propose said proposal. For more complicated proposals however, the organisation might not explicitly state their position but for example only list amendments. In addition, the Have Your Say platform does not include whether an actor supports a proposal or proposes amendments, but only displays the feedback.