\chapter*{One-page executive summary\hfill} \addcontentsline{toc}{chapter}{One-page executive summary}

The relevance of interest groups in the context of European decision-making is evergrowing. Naturally, this evolution is accompanied by a growing corpus of texts put forward by these various interest groups, each and every one of them aiming to make their mark on the legislative process. On the other hand, institutions such as the European Commission (EC) and European Parliament (EP) depend greatly on technical knowledge as input for their policies, and thus this linkage between said institutions and civil society proves crucial to the democratic workings of the Union. Due to the collaborative nature of interest group politics, it is highly important for lobby groups to assess the political arena surrounding a certain policy proposal. In other words, the group must be able to determine which actors are in favour and which are against the proposal, in an effort to form meaningful and advantageous alliances and be able to impose their point of view.

The main goal of this dissertation is to provide a way to carry out a quantitative content analysis of the corpus of published texts within the EU’s political context. This process however, is not as straightforward as it may seem and three issues pop up. Firstly, the relevant actors need to be identified. The mapping out of the  actors involved in EU legislative processes has recently been made possible through the introduction of the Have Your Say platform by the EU Commission, which enables the public to express their opinion at all stages of the policy making process. Secondly, current applications of quantitative text analysis with regards to EU lobbying have used the complete text of a position document to estimate one policy dimension only. However, EU policy making is characterised by policy spaces that have more than one dimension.  Thirdly, it is difficult to predict which policy outcome is prefered by each actor. While for some proposals the position of the actor is clear cut, this will not necessarily be evident in more multifaceted proposals.

One of the core activities of an association is to provide information regarding proposals to its members. For this, most associations have set up some sort of software platform that allows them to share documents and organise meetings. Ideally, the members of an association can access all the association’s information from a single source. In the case of APPLiA Europe, an API web server was constructed as part of this dissertation, to efficiently provide data for the internal members platform. Named Entity Recognition was performed by using the Google cloud Natural Language AI API. The results of this analysis were also included in the database model. Moreover, Sentiment Analysis was performed using the Vader package. The results of this analysis were added to this database model as well, so that they could additionally be included in the web service. To further aid with the analysis of the policy space, the results of the sentiments analysis are visualised in a scatter plot. The plot shows how much of the text has a positive sentiment and how much has a negative sentiment. This makes easy clustering of the mapped out actors possible using the graphical representation of the sentiments.
