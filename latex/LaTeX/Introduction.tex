\mainmatter
\pagestyle{headings}

\chapter{Introduction}

Between the 1980s and the mid-2000s, the number of interest groups attempting to make their mark on the European Unions’ (EU) policy development process, has quintupled, which underlines the growing relevance of interest group politics \autocite{coen2009}. This evolution has been accompanied by an expanding corpus of texts put forward by the various interest groups that clarify their position, in addition to formulating recommendations on their behalf.

\section*{Interest group politics}
Interest groups can be seen as an umbrella definition for the many actors that are involved in the process of activities aimed at influencing public policy making \autocite{europeancouncil2022}. There are several types of interest groups active at the EU level: NGOs, think tanks, scientific institutes, industry associations, consumer rights groups, external consultants, national authorities, etc. To complicate matters further, not only interest groups that operate on the EU level get involved in the lobbying process, but groups that focus on the national level participate in the process as well, as the EU’s activities have considerable impact on the national level, too.
Within the context of the European Union, the relevant institutions, such as the European Commission (EC)  and European Parliament (EP), heavily rely on technical knowledge as input for their policy-making decisions. It is precisely through this connection between European legislators and the interest groups that a participative democracy takes place, whereby a linkage between EU institutions and the European civil society arises.

As mentioned above, interest groups aim to impact political decision-making.  For this purpose, they dispose of several resources such as money, political support, and most importantly expertise and information \autocite{crombez2022}. By supplying policymakers with the knowledge they possess, for instance on market conditions and certain technological advances, they are able to facilitate the decision-making process by shedding light on the feasibility of a proposal \autocite{dur2008}. Considering the fact that the European Commission tends to have an ``extensive policy agenda and limited policy resources'', the information supplied by interest groups becomes all the more essential.

To influence political decisions, interest groups such as associations representing particular industries, often join their efforts with other groups lobbying for the same cause. They do so of course because as a result, they can put more pressure on political actors due to their increased size. Moreover, collaborating with like-minded lobby groups ensures more information transfers can take place, which is one of the main ways EU institutions rely on interest groups. Thus, in this context, it is of vital importance to properly identify the different groups that could potentially have an interest in a particular political issue. One often-used, simplified manner of identifying the relevant interest groups and their stance on a particular issue is based on the Median Voter Theorem which was first put forward by the Scottish economist Duncan Black in 1948. An essential element of this theory is the assumption that voters and policies can be distributed along a one-dimensional spectrum representing political stances on a particular issue \autocite{black1948}. The theorem then posits that politicians who aim to win elections tend to converge to the point of view of the median voter on the abovementioned one-dimensional axis. This is the case, because voters vote for the person or party that most closely represents their own political viewpoint. Therefore, by assuming a position close to the median voter’s position, the politician is able to attract more voters than if he or she were to adjust their policies more to the left or right of the median voter position.

When deciding on whether to attempt to influence an issue, it is wise for a particular interest group to try to place the relevant actors on the abovementioned axis. Subsequently, several situations can arise. Firstly, it can become clear that most actors are located somewhere near the median of the axis. In this case, no matter the position of the particular interest group at hand, the group will not attempt to collaborate. Indeed, say the interest group is far more extreme, i.e. is positioned far from the median on the one-dimensional axis representing the political positions on the issue, there would be no point in investing time and effort into potential partnerships, as per the Median Voter Theorem, politicians will adjust their policies according to the view of the median voter. On the other hand, if the interest group finds themselves located near the median, it would again be pointless to cooperate with other actors, as the desired outcome will manifest itself either way.

Secondly, the voter landscape could also be much more heterogeneous, in which case the topic of potential cooperation becomes much more relevant. It then comes down to figuring out where the different actors are located on the one-dimensional axis. Subsequently, it is important to determine which coalitions would be most useful in shaping the political outcome and to determine which opponents should be monitored more closely.

\section*{APPLiA Europe}
APPLiA is a Brussels-based trade association that provides a single, consensual voice for the home appliance industry in Europe \autocite{applia2022}. Its members manufacture a wide variety of home appliances: large appliances such as refrigerators, freezers, ovens, dishwashers, washing machines and dryers; small appliances such as vacuum cleaners, irons, toasters and toothbrushes; and heating, ventilation and air conditioning appliances such as air conditioners, heat pumps and local space heaters. The role of the association is two-fold, due to its position between EU institutions and the APPLiA members. Firstly, the association aims to provide information on the activities of the EU commission and the EU parliament on matters that are considered to be relevant for the appliance industry. A very clear example of this are the recent attempts to take regulatory action at the EU level to enforce the placement of microplastic filters in washing machines. Seeing as the textile industry accounts for between 15\% and 30\% of the total amount of plastics entering our oceans, these filters could prove to be a significant step forward \autocite{ries2022}. Secondly, APPLiA aids its members in finding common ground so that a consensus can be reached, in order to represent the voice of the industry at the EU level. As has been laid out previously, to achieve this, it is important to determine the position of the association compared to other actors. This, however, is no easy task, as will become clear in the problem description of this dissertation.

In conclusion, analysing the political arena in which the association operates is essential in order to be effective. Investigating the policy preferences of other actors therefore is an essential part of the association’s core business. Automating the way this information is gathered and presented would mean APPLiA employees would be able to spend more time on formulating concrete recommendations for the sector, because they would not need to focus on manually visiting and scanning all relevant websites that could potentially contain useful information on actor’s positions.

 Not only would the automation of these tasks save APPLiA precious time, but presenting this information in the internal member’s area would also mean less time has to be dedicated to informing the members on the findings, because they would easily be able to find the information themselves and do not necessarily need to rely on APPLiA’s  Secretariat for these tasks.

Therefore, this dissertation will focus on the automatic gathering and analysis of the positions of actors in the EU’s political arena. As will become apparent in the next section, this is not a straightforward task. After the problem description, the methodology of the project will be explained. In what follows thereafter,  the results of the analysis will be discussed. In the last section, the project will be evaluated and recommendations for the future will be listed.
