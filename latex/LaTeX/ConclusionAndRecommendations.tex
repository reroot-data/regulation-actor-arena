\chapter{Conclusion and recommendations}
Given the ever growing amount of documents that are produced in and around the EU’s official documents, the need  for an automatic way of filtering relevant texts and deducting the actor’s position grows. Therefore, we build a system in which texts surrounding a certain proposal are grouped and automatically analysed. The time spent on analysing all relevant positions is one of the most time consuming activities within an association, so any sort of automation is very much welcomed.

We set out to create a system that categorised the texts submitted to the Have Your Say platform. In order to achieve this, three problems were identified. A first objective was to collect and show all relevant texts to colleagues and members. This is exactly what the Have Your Say platform tries to achieve, but this project has tried to further improve on the foundations of the Have Your Say platform. Collecting this data ourselves and developing an API web service allowed us to integrate the information in our internal members system. Not only can relevant texts be selected and non-relevant discarded, new ways of filtering can also be included. Second, a first step was taken towards automatically identifying the different topics that were included in a text. Named Entity Recognition was performed on the text to allow a quick overview of the content of the text. A third and final objective was to automatically determine the sentiments of the text. For this, Sentiment Analysis was performed on the text using the Vader package.

To get to these results, many hurdles were identified along the way. In short, text is hard to analyse. Besides the more general issues of spelling errors and lemmatization, the presence of multiple topics within a single position paper complicates the analysis. In many cases, proposals have many different consequences, so interest groups do not always focus on the same topics. Moreover, interest groups can decide not to make their position too clear, but instead create a vague text on purpose in order to allow them more freedom in the rest of the legislative process.

Certainly, there is still a long way to go before it would be possible to predict the outcome of a policy proposal based on the submitted texts. Therefore, it was decided to create a system that facilitates working on a certain proposal instead of making automatic estimations of the process outcome. Even when the system is not able to determine outcomes, it can still make the process more efficient and provide useful feedback for the association’s functioning.

One limitation of the Have Your Say platform is that the feedback does not include the policy preference of the actors. Since there is no information regarding the outcome variable, unsupervised analysis methods need to be used. While it would be preferential to also explore supervised models, this cannot easily be achieved. As mentioned before, extracting an interest group’s policy preference can be difficult from a single text alone. Therefore, adding policy preferences cannot be easily outsourced on Amazon’s Mechanical Turk or similar platforms and requires in other words specialist knowledge.

Also the used techniques could also further be improved. Two different improvements were identified here. Firstly, not all text that is produced is included in the have your say feedback. The platform includes the submitted documents in the consultation phase of a proposal, which is a very good place to start the analysis. In the future, other sources can be scraped to include even more text in the analysis. Whether it be communications posted on the association’s websites or on specialist media outlets, these texts could also be added to the analysis. Of course, legal implications need to be identified before they can be added.

Also, different lexicons could be considered where possible. The very specific nature of the feedback documents makes it possible to utilise specific lexicons where they exist. Google Cloud offers a service called Classifying Content that works similarly to Named Entity Recognition where custom lexicons can be utilised. It might be beneficial to construct a custom lexicon that contains all relevant entities for one specific association.
